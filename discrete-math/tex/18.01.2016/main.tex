\documentclass[a4paper, 12pt]{article}
\usepackage{cmap}           % Пакет для поиска в полученной пдфке
\usepackage[utf8]{inputenc} % Ззамена кодировки файла на utf8
\usepackage[T2A]{fontenc}   % Подключение кодировки шрифтов
\usepackage[russian]{babel} % Использование русского языка 
\usepackage[left=2cm, right=2cm, top=1cm, bottom=2cm]{geometry} % Изменение размеров полей
\usepackage{indentfirst}    % Красная строка в начале текста
\usepackage{amsmath, amsfonts, amsthm, mathtools, amssymb, icomma, units, yfonts}
\usepackage{amsthm} % Пакет для нормального оформления теорем
\usepackage{graphicx}
\usepackage{tikz}
\usepackage{esvect}
\usetikzlibrary{calc,matrix}

%Теоремы
%11.01.2016
\newtheorem*{standartbase}{Теорема о стандартном базисе}
\newtheorem*{fulllemma}{Лемма}
\newtheorem*{sl1}{Следствие 1}
\newtheorem*{sl2}{Следствие 2}
\newtheorem*{monotonousbase}{Теорема о монотонном базисе}
\newtheorem*{scheme}{Утверждение 1}
\newtheorem*{n2}{Утверждение 2}
\newtheorem*{zhegalkin}{Теорема Жегалкина}
\newtheorem*{poste}{Теорема Поста}

%18.01.2016
\newtheorem*{on2n}{Теорема}
\newtheorem*{o2ndivn}{Теорема}
\newtheorem*{existsFgthen2ndivn}{Теорема}

\renewcommand{\qedsymbol}{\textbf{Q.E.D.}}

\begin{document}
\title{Дискретная математика. Модуль 3. Лекция 1}
\author{Лекторий ПМИ ФКН 2015-2016\\Бубнова Валерия\\Жижин Пётр\\Пузырев Дмитрий}
\date{18 января 2016}

\maketitle
\section{Размер схемы. Сложность булевой функции. Верхние и нижние оценки сложности}
\subsection*{Размер схемы. Сложность булевой функции}
\textit{Размер булевой схемы} --- это количество присваиваний в схеме 
$g_1, \ldots, g_L$ для вычисления функции $f: \{0, 1\}^n \rightarrow \{0, 1\}$.

\textit{Сложность функции f в базисе B} --- это минимальный размер булевой схемы, 
вычисляющей функцию $f$ в базисе B. Если базис не указывают -- имеют в виду 
стандартный базис $\{\lnot, \lor, \land\}$. \textit{Обозначение:} $C(f)$.

\underline{Утверждение:} Если $B$ -- конечный базис, тогда $\exists c: \forall f $ если $f$ 
вычисляется схемой $S$ в $B$ размера $L$, тогда существует схема в стандартном базисе
размера меньше, чем $c \cdot L$, вычисляющая ту же функцию $f$. 
\begin{proof}
    Пусть $f$ вычисляется схемой $S$ в базисе $B$ путём следующих присваиваний:
    \[
    g_1, \ldots, g_k, \ldots, g_L = f
    \]
    Рассмотрим некоторую $g_k$ как функцию в базисе $B$ от каких-то аргументов.
    \[
    g_k = g(\ldots), g \in B
    \]
    Для этой функции есть некоторая схема в стандартном базисе некоторого размера $L_k'$
    (так как в стандартном базисе любая функция вычислима, в том числе и $g$).
    
    Каждую $g_i$ заменим на соответствующую схему в стандартном базисе размера $L_i'$.
    Тогда и вся функция вычислима в стандартном базисе схемой размера:
    \[
    L'= L_1' + L_2' + L_3' + \ldots + L_L' \leqslant L \cdot max(L_1', L_2', \ldots
    L_L') = L \cdot c_f
    \]

    Для того чтобы теперь подобрать константу $c$ в определении опять возьмем наибольшее
    из всех $c_f$.
\end{proof}

\subsection*{Верхняя оценка схемной сложности}
\begin{on2n}
    $C(f) = O(n \cdot 2^n)$
\end{on2n}
\begin{proof}
    Повторим предыдущие рассуждения при доказательстве того, что в стандартном
    базисе любая функция вычислима. Для этого вспомним:
    \[
    f(x) = \bigvee\limits_{\substack{a: f(a) = 1 \\ a \in \{0, 1\}^n}} x^a, \ 
    x^a = \bigwedge\limits_{i = 1}^{n} x_i^{a_i}
    \]
    Нетрудно посчитать, что схема для вычисления $x^a$ имеет размер $L_a = O(n)$.
    Тогда итоговый размер схемы $L \leqslant 2^n \cdot O(n) \iff L = O(n \cdot 2^n)$.
\end{proof}
\begin{o2ndivn}
    $C(f) = O(\frac{2^n}{n})$
\end{o2ndivn}
\begin{proof}
    В доказательстве много возни, желающие могут найти и прочитать.
    Идея в том, что многие схемы можно повторять примерно $n^2$ раз.
\end{proof}

\subsection*{Нижняя оценка схемной сложности}
\begin{existsFgthen2ndivn}
    Существует функция $f: \{0, 1\}^n \rightarrow \{0, 1\}$
    такая, что $C(f) \geqslant \frac{2^n}{10n}$ 
    (в точности то же самое, что $C(f) = \Omega(\frac{2^n}{n})$).
\end{existsFgthen2ndivn}
\begin{proof}
    Воспоьзуемся мощностным методом. Всего булевых функций от $n$ аргументов $2^{2^n}$.

    Теперь узнаем, сколько булевых схем размера меньше либо равных некоторого фиксированного
    числа $L$. Для этого будем кодировать схемы двоичными словами. Посмотрим на 
    какое-то присваивание в схеме $S$: $g_k = g(g_i, g_j)$. Для кодирования
    самой функции $g$ нужно 2 бита (так как в стандартном базисе всего три функции).
    Для кодирования номеров аргументов $i$ и $j$ нужно битов не более, чем $log_2L$.
    А значит для всего присваивания $g_k$ нужно не более $2 \cdot (1 + log_2L)$ бит.

    Итого размер одной схемы в битах: $L \cdot 2 \cdot (1 + log_2L)$. Каждая схема
    кодирует ровно одну функцию. А значит каждое двоичное слово кодирует не более 
    одной функции (так как некоторые двоичные слова ни одну схему не задают).

    Получается и схем размера $L$ не более, чем двоичных слов для схем такой длины,
    то есть: $2^{2L(1 + log_2L)}$.

    Возьмем $L = \frac{2^n}{10n}$. Размер схемы в битах тогда будет равен: 
    \[
    L_2 = \frac{2^n}{10n} \cdot 2 \cdot \left(1 + log_2\left( \frac{2^n}{10n} \right)\right) = 
    \frac{2^n}{5n} \left( 1 + n - log_2(10n) \right), 1 - log(10n) \leqslant 0 \implies
    L_2 \leqslant \frac{2^n}{5n}\cdot n = \frac{2^n}{5}
    \]
    А значит функций, задающейся схемой такой длины не более чем $2^{\frac{2^n}{5}}$.
    Нетрудно заметить, что это число значительно меньше числа функций от $n$ аргументов.
    \[
        2^{\frac{2^n}{5}} < 2^{2^n}
    \]

    А значит существует функция, задающаяся схемой длины больше, чем $L$.
\end{proof}

Всё очень хорошо, но можно ли задать такую функцию явно? К сожалению, с этим есть
некоторые проблемы. В 1984 году нашли $f$ такую, что $C(f) \geqslant 3n$.
Прорывом 2015 стала функция $f$ такая, что 
$C(f) \geqslant \left( 3 + \frac{1}{86} \right)n$.

\subsection*{Схемы для сложения и умножения двоичных чисел. Схема для проверки графа на связность}



\end{document}
