\documentclass[12pt,a4paper]{article}
\usepackage{listings}
\usepackage{amssymb,amsmath,mathtools,amsthm}
\usepackage[utf8]{inputenc}
\usepackage[russian]{babel}
\usepackage[top=0.5in, bottom=0.75in, left=0.625in, right=0.625in]{geometry}
\usepackage{graphicx}
\usepackage{listings}
\usepackage{algorithm}
\usepackage[noend]{algpseudocode}
\usepackage{indentfirst}
\usepackage[colorlinks=true, urlcolor=magenta]{hyperref}
\usepackage{pgfplots}
\usepackage{forest}

\title{Алгоритмы и структуры данных, лекция 3}
\date{19.01.2016}
\author{}

\begin{document}
\maketitle

$\Theta(g(n)) = \left\{ f(n)\mid \exists c_1>0, c_2>0 \exists n_0: \forall n \geqslant n_0 \implies 0\leqslant c_1g(n)\leqslant f(n) \leqslant c_2g(n) \right\}$

$O(g(n)) = \left\{ f(n)\mid \exists  c_2>0 \exists n_0: \forall n \geqslant n_0 \implies 0\leqslant f(n) \leqslant c_2g(n) \right\}$

$\Omega(g(n)) = \left\{ f(n)\mid \exists c_1>0 \exists n_0: \forall n \geqslant n_0 \implies 0\leqslant c_1g(n)\leqslant f(n) \right\}$

$o(g(n)) = \left\{ f(n)\mid \forall  c_2>0 \exists n_0: \forall n \geqslant n_0 \implies 0\leqslant f(n) \leqslant c_2g(n) \right\}$

$\omega(g(n)) = \left\{ f(n)\mid \forall c_1>0 \exists n_0: \forall n \geqslant n_0 \implies 0\leqslant c_1g(n)\leqslant f(n) \right\}$

$\lambda n.n \in o(\lambda n.n \log_2 n)$

$\log_c n = \frac {\log_2 n}{\log_2 c}$ --- поэтому не пишут основание логарифма.

Вернёмся к сортировке слиянием. 

Алгоритм состоит из трёх шагов:

\vspace{0.5cm}
\textbf{Разделяй и властвуй}
\begin{enumerate}
    \item Разбить задачу на подзадачи; | $\Theta(1)$
    \item Рекурсивно решить подзадачи; | $2T(\frac n2)$
    \item Объединить решения задач в одно глобальное. | $\Theta(n)$
\end{enumerate}

$T(n) = 2T(\frac n2) + \Theta(n)$

$\Theta(n\log n)$

\section*{Быстрая сортировка}

Задача та же --- отсортировать массив.

Воспользуемся методом ``Разделяй и властвуй''. Разобьем по-другому:

Выберем в массиве опорный элемент $x$ (как-то). Пройдем по всем элементам и запишем те элементы, что меньше $x$ до него, а те, что больше --- после.

Две подзадачи: сортировка двух подмассивов.

Третий шаг --- соединить их.

\begin{lstlisting}
Partition(a, p, q)
    i := p
    for j - p+1 to q do
        if a[j] < a[p] then
            i := i+1
            a[i], a[j] = a[j], a[i]
            return i
\end{lstlisting}

\begin{enumerate}
    \item Разбить задачу на подзадачи; | $\Theta(n)$
    \item Рекурсивно решить подзадачи; | $T(r-1) + T(n-r)$
    \item Объединить решения задач в одно глобальное. | $0$
\end{enumerate}

$T(n) = T(r-1) + T(n-r) + \Theta(n)$

Оптимальный вариант:

$T(n) = T(\frac n2) + T(\frac n2) + \Theta(n) \implies \Theta(n\log n)$

Худший вариант:

$T(n) = T(\frac n-1) + \Theta(n) \implies \Theta(n^2)$

Пусть при каждом разбиении обе части не менее четверти.

Подзадача типа j:
Размер входного массива $n'$; $\left( \frac14 \right)^j < n' \leqslant\left( \frac34 \right)^j$

Время на подзадачу типа $j$ --- $O\left( \left( \frac34 \right)^j \right)n$

Заметим, что подзадачи типа $j$ не пересекаются; из задачи типа $j$ получаются задачи типа не меньше $j+1$.

Количество подзадач типа $j$ $\leqslant \left( \frac43 \right)^{j+1}$.

Время на все подзадачи типа $j$ --- $O\left( \left( \frac34 \right)^jn\cdot\left( \frac43 \right)^{j+1} \right) = O(n)$;

Максимальное $j$ --- $\log_{\frac43}n$

Общее время --- $O(n\log n)$


Но это если нам везёт.

\section*{Как обеспечить везение?}

Мы хотим, чтобы опорный элемент был близок к середине (в отсортированном массиве). Условно, в пределах средних двух четвертей. Если выбирать наугад, вероятность $50\%$.

Предположим, что мы выбираем случайный элемент. Распределим все прочие, и если одна из частей меншь четверти, забудем про этот элемент и выберем другой. Повторим, пока не получим хороший элемент. В среднем на это уйдёт две попытки $\left( \frac1P \right)$. На сложности алгоритма это нье сказывается никак, т.к. меняется только константа. Зато теперь так не только в лучшем, но и в среднем случае.
\end{document}
